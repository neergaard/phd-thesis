%*******************************************************
% Abstract
%*******************************************************
%\renewcommand{\abstractname}{Abstract}
\pdfbookmark[1]{Resumé}{Resumé}
\addcontentsline{toc}{chapter}{\tocEntry{Resumé}}
\begingroup
\let\clearpage\relax
\let\cleardoublepage\relax
\let\cleardoublepage\relax

\chapter*{Resumé}
\begin{otherlanguage}{danish}

Søvnforstyrrelser er udbredte i samfundet og har konsekvenser for personlig sundhed og dødelighed, herunder øget risiko for hjertekarsygdomme, metaboliske og psykiatriske komplikationer.
Derudover er søvnforstyrrelser en stor omkostning for samfundet, og det er anslået at komplikationer følgende søvnforstyrrelser bidrager til flere hundrede milliarder dollars om året. 

Den nuværende standard for diagnose af søvnforstyrrelse og søvnsygdomme er baseret på manuel analyse af elektroencephalografi (EEG), elektrooculografi (EOG), eletromyografi (EMG) og cardio-respiratoriske variable registreret under søvn efter et sæt retningslinjer defineret af American Academy of Sleep Medicine. 
Flere studier har imidlertid vist, at dette introducerer et subjektivt bias i den resulterende søvnanalyse på grund af fortolkninger af scoringsretningslinjer, forskelle i optageudstyr og -opsætning blandt forskellige søvnklinikker samt tilstedeværelsen af søvnforstyrrelser, der negativt påvirker normale søvnmønstre.

Hovedformålet med denne afhandling er at udvikle et system baseret på kunstig intelligens, der kan hjælpe klinikere i analysen af søvnmønstre. Dette realiseres i tre forskningstemaer, der hver især fokuserer på et separat aspekt af søvnanalyse.

Først præsenteres resultater relateret til udvikling af metoder til automatisk klassificering af søvnstadier, som er en vigtig del af analysen af søvnmønstre.
To modeller baseret på dybe neurale netværk anvendt på EEG, EOG og EMG signaler blev udviklet af denne grund.
Ved hjælp af de rå signaler fra 14086 polysomnografier (PSG) til klassificering af søvnstadier opnåede den første model en gennemsnitlig nøjagtighed på 86.9\% målt over 1584 PSG'er samlet fra fem uafhængige datasæt.
Den anden model brugte krydskorrelations-repræsentationer af signaler fra 2784 PSG'er til klassificering af søvnstadier med en nøjagtighed på 86.8\% målt henover 70 PSG'er scoret af seks søvnteknikere.

Dernæst præsenteres en metode til automatisk detektion af mikro-opvåg-ninger, benspjæt, og apnø-lignende perioder.
En model blev designet baseret på dybe neurale netværk anvendt på EEG, EOG, EMG og åndedrætssignaler.
Modellen var i stand til at lokalisere og klassificere begivenheder i data og blev testet på mere end 1000 PSG'er.
Desuden blev modellen brugt til overførselslæring, hvor en finjusteret optimeringsstrategi af modellen effektivt kunne gendanne tabt ydeevne forårsaget af et reduceret sæt inputkanaler.
En fleksibel model som denne vil være nyttig i en klinisk sammenhæng.

Det sidste forskningstema vedrørte klassificering af søvnforstyrrelser ved hjælp af kunstig intelligens.
En model blev designet baseret på repræsentationer af hypnodensiteten og probabilistiske algoritmer til adskillelse af narkolepsi type 1 (NT1) patienter fra kontroller og patienter med andre centrale hypersomnier.
NT1 blev identificeret med 91\% sensitivitet og 96\% specificitet i testprøven, mens replikation i to uafhængige datasæt gav tilsvarende resultater.

Samlet set præsenterer denne afhandling nye automatiske metoder til klinisk søvnanalyse baseret på kunstig intelligens.
Sammenlignet med de nuværende analyse-standarder kan de foreslåede modeller markant reducere analysetiden i kraft af at være hurtigere end manuel analyse, samtidig med at de kan levere lignende eller højere ydeevne.

\end{otherlanguage}
\endgroup
