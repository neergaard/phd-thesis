%*******************************************************
% Acknowledgments
%*******************************************************
\pdfbookmark[1]{Acknowledgments}{acknowledgments}
\addcontentsline{toc}{chapter}{\tocEntry{Acknowledgements}}

% \begin{flushright}{\slshape
%     We have seen that computer programming is an art, \\
%     because it applies accumulated knowledge to the world, \\
%     because it requires skill and ingenuity, and especially \\
%     because it produces objects of beauty.} \\ \medskip
%     --- \defcitealias{knuth:1974}{Donald E. Knuth}\citetalias{knuth:1974} \citep{knuth:1974}
% \end{flushright}



\bigskip

\begingroup
\let\clearpage\relax
\let\cleardoublepage\relax
\let\cleardoublepage\relax
\chapter*{Acknowledgments}
% \vspace{3cm}
\vfill

First of all, I would like to express my deepest gratitude to my main supervisor \emph{Helge B. D. Sørensen}, and clinical co-supervisors \emph{Poul Jennum}, and \emph{Emmanuel Mignot}, for believing in me and allowing me to work on something that I truly enjoy. 
Thank you for your guidance and support during the last three and a half years.

I would also like to thank all the people associated with the Sleep Lab at Stanford University for welcoming me into their impressive scientific environment, in particular \emph{Eileen Leary}, \emph{Hyatt Moore}, and \emph{Aditya Ambati}; a big thanks to \emph{Stephanie Lettieri} for all your help in general; and also my fellow visiting student researcher \emph{Stanislas Chambon} for engaging and enlightening discussions about machine learning, and for convincing me to switch to PyTorch.

Thank you to all my colleagues at the Section for Health Technology for interesting scientific discussions in our biweekly meetings; my fellow PhD students \emph{Matteo Cesari}, \emph{Mads Olsen}, \emph{Umaer Hanif}, \emph{Andreas Brink-Kjær}, and \emph{Søren Møller Rasmussen} for all of our scientific discussions, and for making sure that I get enough coffee.
A particular thanks goes to \emph{Julie A. E. Christensen}.
Thank you for your friendship and for always being willing to offer scientific insights and inputs to manuscripts and projects.

A huge thanks goes out to all of the students that I have been engaged with during my time at Stanford: \emph{Lorenzo}, \emph{Andreas}, \emph{Kristina}, \emph{Rasmus}, \emph{Jonathan}, \emph{Jakob}, \emph{Thomas}, \emph{Vicente}, \emph{Aske}, and \emph{Joakim}; many of whom have become good friends, and some even my fellow PhD students.
I hope you learned as much from me as I learned from you.
Especially, I would like to express my gratitude towards \emph{Jens. B. Stephansen}, who invited me to work with him on the narcolepsy detector, which has been an incredible experience.

I would also like to extend a huge thanks to \emph{Matteo Cesari}, \emph{Martin Nørgaard}, \emph{Sebastian Holst}, \emph{Rasmus Malik Thaarup Høegh}, and \emph{Julie A. E. Christensen}. 
You have provided valuable comments to this thesis.

To all my friends and family: you have been sorely missed, and I can't wait to hang out and spend time with you all again pending no further restrictions due to corona-virus.
A big thank you to my wonderful in-laws, \emph{Annette \& Amr}, for allowing me to set up a make-shift office in your dining room and babysitting Theodor during the corona-crisis.
To my father \emph{Lars-Ulrik}; thank you for always believing in me, for your guidance and our lunches together at DTU. 
% To my mother \emph{Helle}, who is not with us anymore; I miss you and I hope this will make you proud.

And finally, from the bottom of my heart; I want to acknowledge and thank \emph{Alexandra}, the love of my life, for your enthusiasm and willingness to go on an adventure and live in California for almost two years while carrying our first child; for your infinite patience, resilience, love and understanding while I tried my hands at science---without you, this would not have been possible.

% \begin{itemize}
%     \item Supervisors
%     \item Biomedical signal processing group
%     \item Stanford team
%     \item Fiancee
%     \item In-laws for letting me use their dining room as a temporary office during the coronavirus outbreak.
% \end{itemize}

% Many thanks to everybody who already sent me a postcard!

% Regarding the typography and other help, many thanks go to Marco
% Kuhlmann, Philipp Lehman, Lothar Schlesier, Jim Young, Lorenzo
% Pantieri and Enrico Gregorio\footnote{Members of GuIT (Gruppo
% Italiano Utilizzatori di \TeX\ e \LaTeX )}, J\"org Sommer,
% Joachim K\"ostler, Daniel Gottschlag, Denis Aydin, Paride
% Legovini, Steffen Prochnow, Nicolas Repp, Hinrich Harms,
% Roland Winkler, Jörg Weber, Henri Menke, Claus Lahiri,
% Clemens Niederberger, Stefano Bragaglia, Jörn Hees,
% Scott Lowe, Dave Howcroft, Jos\'e M. Alcaide, David Carlisle,
% Ulrike Fischer, Hugues de Lassus, Csaba Hajdu, Dave Howcroft, 
% and the whole \LaTeX-community for support, ideas and
% some great software.

% \bigskip

% \noindent\emph{Regarding \mLyX}: The \mLyX\ port was intially done by
% \emph{Nicholas Mariette} in March 2009 and continued by
% \emph{Ivo Pletikosi\'c} in 2011. Thank you very much for your
% work and for the contributions to the original style.


\endgroup
