\section{Chapter summary}\label{sec:sleep-stage-classification:summary}

Sleep%
\graffito{\ref{hypothesis:sleep-stages}: \hypothesis{} sleep stages} %
stage classification is performed manually by experts in sleep clinics leading to major inter- and intra-variability~\cite{Norman2000,Rosenberg2013,Younes2017,Younes2016,Younes2018}.
One potential way to overcome this challenge is to assist or augment the manual scoring with fully automatic intelligent systems (\ref{hypothesis:sleep-stages}), that provide consistency and robustness in the analysis of sleep patterns.
In this chapter, we introduced methods for automating sleep stage classification using deep neural networks with two separate model frameworks to answer research questions~\ref{question:sleep-stages-classification}, \ref{question:sleep-stages-datasets}, and \ref{question:sleep-stages-guarantee}.

\Cref{sec:paperi}\graffito{\ref{question:sleep-stages-classification}: \questionSleepStageClassification} described the initial version of the \ac{MASSC} algorithm, and end-to-end deep learning model based on the ResNet-50 architecture. 
We trained and tested the algorithm on a collective of 2310 \acp{PSG} using three different training strategies, the best of which yielded a high accuracy value of 84.1\% and a \cohen of 0.746.
In view of the large number of \ac{PSG} recordings included in the study, these numbers compare favorably to the current state-of-the-art in automatic sleep stage scoring, as well as the reported inter-rater reliability measures described in~\cref{sec:challenges-sleep-stage-scoring}.

However%
\graffito{\ref{question:sleep-stages-datasets}: \questionSleepStageDatasets}, %
like many other published papers on automatic sleep stage classification, the results reported in~\cref{sec:paperi} are based solely on a single cohort of \acp{PSG}, which immediately raises concerns over the actual generalizability of the model.
In~\cref{sec:paperii} we applied an updated version of the \ac{MASSC} algorithm in four different experimental settings using five cohorts differing in size, demographics, inherent co-morbidities and recording setups.
We found that training models on individual cohorts yielded large variations in classification performance both in \ac{LOCI} and \ac{LOCO} training configurations. 
Strikingly, we found consistently higher sleep stage classification accuracy as a function of the data fraction by mixing cohorts in the training data compared to training models on single cohorts.
Using 100\% of the training data, our model achieved an accuracy of 86.9\% and a \cohen of 0.799, which in light of the high numbers of both training and testing records compares favorably to the state-of-the-art, as well as our previous reported results in~\cref{sec:paperi}.

The%
\graffito{\ref{question:sleep-stages-guarantee}: \questionSleepStageGuarantee} %
final section described the sleep stage classification part of the \ac{STAGES} model.
Here, we used specific transformations of the input \ac{PSG} signals coupled with multiple realizations of a deep neural network architecture to create a final ensemble model for classifying sleep stages.
Based on a total 2784 \acp{PSG}, the best performing model as determined by a \(2^5\)-factorial experimental design yielded an accuracy of 86.8\% on a dataset scored by six technicians, while outperforming every single one based on both a biased and unbiased consensus score.
The model was also shown to be stable with respect to the presence of several sleep disorders, with the exception of narcolepsy which had a significant impact on the algorithm.\graffito{The implications of this will be detailed in~\cref{chap:classification-sleep-disorders}}