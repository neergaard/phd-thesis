\section{Chapter summary}\label{sec:eventdetection-summary}
This%
\graffito{\ref{hypothesis:sleep-events}: \hypothesis{} sleep events.} %
chapter concerned the detection of sleep events motivated by research hypothesis~\ref{hypothesis:sleep-events}, and research questions~%
\ref{question:sleep-events-detection}, %
\ref{question:sleep-events-transfer}, and %
\ref{question:sleep-events-jointly}%
\graffito{\ref{question:sleep-events-detection}: \questionSleepEventsDetection.}%
\graffito{\ref{question:sleep-events-transfer}: \questionSleepEventsTransfer.}%
\graffito{\ref{question:sleep-events-jointly}: \questionSleepEventsJointly.}.

The first study described the \ac{MSED} model.
Here, the main objective was to detect multiple sleep events in \ac{PSG} studies simultaneously and independently using only a single model, which was trained and tested on 1485 and 1000 \acp{PSG}, respectively. 
Optimal arousal detection was obtained by including a recurrent neural network module and using a dynamic default event window yielding an F1 score of 0.75, while optimal leg movement detection was obtained with a static window yielding an F1 score of 0.65.

The second study in this chapter presented the application of the \ac{MSED} algorithm for arousal detection under the channel mismatch problem.
We investigated two deep transfer learning strategies for overcoming the channel mismatch problem for cases, where two datasets do not contain exactly the same setup leading to degraded performance in single-\ac{EEG} models. 
Using a fine-tuning strategy, the model yielded similar performance to the baseline model (\(\mathrm{F1}=0.68\) and \(\mathrm{F1}=0.69\), respectively), and was significantly better than a comparable single-channel model.
While these results are promising, they will have to be validated in a larger setting across separate cohorts.

Motivated by the preliminary results obtained previously, we investigated whether the \ac{MSED} model could be extended with more input channels for added detection of sleep disordered breathing events, the results of which are shown in the last study of this chapter.
We tested different variations on network architecture and found that a split-stream network with each stream responsible for separate sets of input channels was beneficial for our task.
The results from evaluating on 1000 separate \ac{PSG} recordings was F1 scores of \numlist{0.70;0.63;0.62} for \ac{Ar}, \ac{LM}, and \ac{SDB} detection, respectively.
Interestingly, we also found that the model performed better with respect to F1 score for each separate class, when detecting events jointly instead of using single models for each class.
However, this model remains severely understudied, and future efforts should concentrate on including more data sources, evaluating on a dataset containing multiple scorers, and benchmarking against state of the art methods.



    % Aspects that were not investigated in this thesis, but are of interest in future work include
    % \begin{itemize}
    %     \item An evaluation of individual sleep events when placed in the context of a micro-sleep architecture, \eg a high-resolution hypnodensity.
    % \end{itemize}