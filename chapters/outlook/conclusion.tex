\chapter{Conclusion}\label{chap:conclusion}
\begin{flushright}{\slshape 
        Did we learn a lesson here I'm not seeing?} \\ \medskip
        --- Summer Sanchez\\Rick and Morty, season 1, episode 9
\end{flushright}

\vspace{6cm}

The main focus of this thesis was \emph{\objective}.
The application of such a system would benefit clinicians and patients by shifting time spent on analysis towards patient care.
The system was realized in three parts, which each addresses a specific sub-hypothesis:

\bigskip
\noindent \ref{hypothesis:sleep-stages}: \textit{\hypothesis\xspace sleep stages}.

\Cref{chap:sleep-stage-classification} presented the \ac{MASSC} model for automatic classification of sleep stages using raw \ac{EEG}, \ac{EOG}, and chin \ac{EMG} data extracted from \ac{PSG}.
This model was initially proposed in~\cite{Olesen2018c}, where it was trained and tested on 1850 and 230 \acp{PSG}, respectively, yielding an accuracy of 84\%. 
Increasing both the volume and diversity of the data increased performance to 87\%, which was described in~\cref{sec:paperii}.

Also presented was the \ac{STAGES} model for sleep stage classification based on cross-correlation representations of the \ac{EEG}, \ac{EOG} and chin \ac{EMG} and an ensemble of deep neural networks.
This model was trained on 2784 \acp{PSG} and subsequently validated against six technicians on 70 \acp{PSG}, where it outperformed all technicians both on a biased and unbiased consensus score.
The model was furthermore found stable with respect to underlying sleep pathologies in all cases except for narcolepsy, which was exploited in~\cref{chap:classification-sleep-disorders}.

These findings suggest that \ac{AI}-based systems such as deep neural networks can augment clinicians with sleep stage classification, and that volume and diversity in datasets are key to high performance.
% ~\cite{Olesen2020AutomaticSetting}
% \noindent \paragraph{Main outcome :} the \ac{MASSC} model.

\bigskip
\noindent \ref{hypothesis:sleep-events}: \textit{\hypothesis\xspace sleep events}.

\Cref{chap:sleep-event-detection} presented the \ac{MSED} model for sleep event detection.
An initial version of the model was described in~\cref{sec:paperiv} for detection of arousals and leg movements, which was augmented to include sleep disordered breathing events in~\cref{sec:papervi}, using raw \ac{EEG}, \ac{EOG}, chin and leg \ac{EMG}, and respiratory data extracted from the \ac{PSG}.
Training and testing the model on 1653 and 1000 \acp{PSG}, respectively, yielded F1 scores of 0.70, 0.63, and 0.62 for arousal, leg movement, and sleep disordered breathing event detection, respectively.
The performance was higher when detecting events jointly compared to corresponding single-event models.
Index values computed from detected events correlated well with manual annotations with \(r^2\)-values of 0.73, 0.77, and 0.78, respectively.

The \ac{MSED} model was applied in a transfer learning setup under the channel mismatch problem, where the target dataset contained only one \ac{EEG} channel.
This problem has been investigated previously for the case of sleep stage classification, but remains under-investigated in sleep event detection.
Using a network pre-trained on a full montage of input channels, stripped of the input processing layers, and subsequently fine-tuned on a smaller dataset allowed for recovery of F1 performance compared to the full montage model.

% The findings argue that an \ac{AI}-based system for sleep event detection can be of use for clinicians due to the performance level, 

% \noindent \paragraph{Main outcome :} the \ac{MSED} model.

\bigskip
\noindent \ref{hypothesis:sleep-disorders}: \textit{\hypothesis\xspace sleep disorders}.
\cref{chap:classification-sleep-disorders} presented the application of the \ac{STAGES} for narcolepsy detection.
The motivation behind this was the finding in~\cref{sec:paperiii}, which led to the development of a probabilistic model using features derived from the hypnogram representation of a \ac{PSG} and a Gaussian process classification algorithm. 
The \ac{STAGES} model was able to classify \ac{NT1} patients with a 91\% sensitivity and 96\% specificity, which was increased to 99\% by adding \hla typing. 
This was replicated in an independent cohort of data from two continents for an optimized sensitivity and specificity of 94\% and 94\%.
These findings match the current gold standard for narcolepsy diagnosis while having the benefit of using the \ac{PSG} alone or in combination with blood samples.

% \noindent \paragraph{Main outcome :} the \ac{STAGES} model.

\bigskip

Altogether, the findings presented in this thesis underline the practicality and utility of incorporating \ac{AI}-based systems in the sleep clinic by providing fast and accurate analysis of sleep stages, sleep events, and sleep disorders.

% This was based on the hypothesis that \MakeLowercase{\hypothesis} sleep stages, sleep events, and sleep disorders.
% In this thesis, the system was realized using three models.
% This chapter will touch on the results of three models as described in~\cref{chap:sleep-stage-classification,chap:sleep-event-detection,chap:classification-sleep-disorders}, and discuss aspects of including artificial intelligence in sleep clinics.

% \begin{enumerate}
%     \item[\ref{hypothesis:sleep-stages}] \\textit{}
% \end{enumerate}